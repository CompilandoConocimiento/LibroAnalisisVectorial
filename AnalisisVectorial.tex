% ****************************************************************************************
% ************************      ANALISIS VECTORIAL            ****************************
% ****************************************************************************************


% =======================================================
% =======         HEADER FOR DOCUMENT        ============
% =======================================================
    % *********   DOCUMENT ITSELF   **************
    \documentclass[12pt, fleqn]{report}                             %Type of docuemtn and size of font and left eq
    \usepackage[margin=1.2in]{geometry}                             %Margins and Geometry pacakge
    \usepackage{ifthen}                                             %Allow simple programming
    \usepackage{hyperref}                                           %Create MetaData for a PDF and LINKS!
    \usepackage{pdfpages}                                           %Create MetaData for a PDF and LINKS!
    \hypersetup{pageanchor=false}                                   %Solve 'double page 1' warnings in build
    \setlength{\parindent}{0pt}                                     %Eliminate ugly indentation
    \author{Oscar Andrés Rosas & Alan Enrique Ontiveros Salazar}    %Who I am

    % *********   LANGUAJE AND UFT-8   *********
    \usepackage[spanish]{babel}                                     %Please use spanish
    \usepackage[utf8]{inputenc}                                     %Please use spanish - UFT
    \usepackage[T1]{fontenc}                                        %Please use spanish
    \usepackage{textcmds}                                           %Allow us to use quoutes
    \usepackage{changepage}                                         %Allow us to use identate paragraphs
    \usepackage{anyfontsize}                                        %All the sizes

    % *********   MATH AND HIS STYLE  *********
    \usepackage{ntheorem, amsmath, amssymb, amsfonts}               %All fucking math, I want all!
    \usepackage{mathrsfs, mathtools, empheq}                        %All fucking math, I want all!
    \usepackage{centernot}                                          %Allow me to negate a symbol
    \decimalpoint                                                   %Use decimal point

    % *********   GRAPHICS AND IMAGES *********
    \usepackage{graphicx}                                           %Allow to create graphics
    \usepackage{wrapfig}                                            %Allow to create images
    \graphicspath{ {Graphics/} }                                    %Where are the images :D

    % *********   LISTS AND TABLES ***********
    \usepackage{listings, listingsutf8}                             %We will be using code here
    \usepackage[inline]{enumitem}                                   %We will need to enumarate
    \usepackage{tasks}                                              %Horizontal lists
    \usepackage{longtable}                                          %Lets make tables awesome
    \usepackage{booktabs}                                           %Lets make tables awesome
    \usepackage{tabularx}                                           %Lets make tables awesome
    \usepackage{multirow}                                           %Lets make tables awesome
    \usepackage{multicol}                                           %Create multicolumns

    % *********   HEADERS AND FOOTERS ********
    \usepackage{fancyhdr}                                           %Lets make awesome headers/footers
    \pagestyle{fancy}                                               %Lets make awesome headers/footers
    \setlength{\headheight}{16pt}                                   %Top line
    \setlength{\parskip}{0.5em}                                     %Top line
    \renewcommand{\footrulewidth}{0.5pt}                            %Bottom line

    \lhead{                                                         %Left Header
        \hyperlink{chapter.\arabic{chapter}}                        %Make a link to the current chapter
        {\normalsize{\textsc{\nouppercase{\leftmark}}}}             %And fot it put the name
    }

    \rhead{                                                         %Right Header
        \hyperlink{section.\arabic{chapter}.\arabic{section}}       %Make a link to the current chapter
            {\footnotesize{\textsc{\nouppercase{\rightmark}}}}      %And fot it put the name
    }

    \rfoot{\textsc{\small{\hyperref[sec:Index]{Ve al Índice}}}}     %This will always be a footer  

    \fancyfoot[L]{                                                  %Algoritm for a changing footer
        \ifthenelse{\isodd{\value{page}}}                           %IF ODD PAGE:
            {\href{https://compilandoconocimiento.com/yo/}          %DO THIS:
                {\footnotesize                                      %Send the page
                    {\textsc{Oscar Rosas y Alan Ontiveros}}}}       %Send the page
            {\href{https://compilandoconocimiento.com}              %ELSE DO THIS: 
                {\footnotesize                                      %Send the author
                    {\textsc{Compilando Conocimiento}}}}            %Send the author
    }
    
    
    
% ========================================
% ===========   COMMANDS    ==============
% ========================================

    % =====  GENERAL TEXT  ====================
    \newcommand \Quote {\qq}                                        %Use: \Quote to use quotes
    \newcommand \Over {\overline}                                   %Use: \Bar to use just for short
    \newcommand \ForceNewLine {$\Space$\\}                          %Use it in theorems for example
    
    % =====  NEW ENVIRONMENTS  ================
    \newenvironment{Indentation}[1][0.75em]                         %Use: \begin{Inde...}[Num]...\end{Inde...}
    {\begin{adjustwidth}{#1}{}}                                     %If you dont put nothing i will use 0.75 em
    {\end{adjustwidth}}                                             %This indentate a paragraph
    \newenvironment{SmallIndentation}[1][0.75em]                    %Use: The same that we upper one, just 
    {\begin{adjustwidth}{#1}{}\begin{footnotesize}}                 %footnotesize size of letter by default
    {\end{footnotesize}\end{adjustwidth}}                           %that's it

    \newenvironment{MultiLineEquation}[1]                           %Use: To create MultiLine equations
        {\begin{equation}\begin{alignedat}{#1}}                     %Use: \begin{Multi..}{Num. de Columnas}
        {\end{alignedat}\end{equation}}                             %And.. that's it!
    \newenvironment{MultiLineEquation*}[1]                          %Use: To create MultiLine equations
        {\begin{equation*}\begin{alignedat}{#1}}                    %Use: \begin{Multi..}{Num. de Columnas}
        {\end{alignedat}\end{equation*}}                            %And.. that's it!

    % =====  GENERAL MATH  ====================
    \DeclareMathOperator \MegaSpace {\quad \quad}                   %Use: \MegaSpace for a cool mega mega space
    \DeclareMathOperator \Space {\quad}                             %Use: \Space for a cool mega space
    \DeclareMathOperator \MiniSpace {\;}                            %Use: \Space for a cool mini space
    \newcommand \Such {\MiniSpace|\MiniSpace}                       %Use: \Such like in sets
    \newcommand \Also {\MiniSpace \text{y} \MiniSpace}              %Use: \Also so it's look cool
    \newcommand \Remember[1]{\Space\text{\scriptsize{#1}}}          %Use: \Remember so it's look cool
    \newcommand{\abs}[1]{\left\lvert #1 \right\lvert}               %Use: \abs{expression} for |x|
    \newcommand{\Abs}[1]{\left\lVert #1 \right\lVert}               %Use: \Abs{expression} for ||x||

    \newtheorem{Theorem}{Teorema}[section]                          %Use: \begin{Theorem}[Name]\label{Nombre}...
    \newtheorem{Corollary}{Colorario}[Theorem]                      %Use: \begin{Corollary}[Name]\label{Nombre}...
    \newtheorem{Lemma}[Theorem]{Lemma}                              %Use: \begin{Lemma}[Name]\label{Nombre}...
    \newtheorem{Definition}{Definición}[section]                    %Use: \begin{Definition}[Name]\label{Nombre}...

    % =====  CONTAINERS   ====================
    \newcommand{\Set}[1]{\left\{ \MiniSpace #1 \MiniSpace \right\}} %Use: \Set {Info}
    \newcommand{\Brackets}[1]{\left[ #1 \right]}                    %Use: \Brackets {Info} 
    \newcommand{\Wrap}[1]{\left( #1 \right)}                        %Use: \Wrap {Info} 
    \newcommand{\pfrac}[2]{\Wrap{\dfrac{#1}{#2}}}                   %Use: Put fractions in parentesis

    % =====  LOGIC  ==========================
    \DeclareMathOperator \doublearrow {\leftrightarrow}             %Use: \doublearrow for a double arrow
    \newcommand \lequal {\MiniSpace \Leftrightarrow \MiniSpace}     %Use: \lequal for a double arrow
    \newcommand \linfire {\MiniSpace \Rightarrow \MiniSpace}        %Use: \lequal for a double arrow
    \newcommand \longto {\longrightarrow}                           %Use: \longto for a long arrow

    % =====  FAMOUS SETS  =====================
    \DeclareMathOperator \Naturals     {\mathbb{N}}                 %Use: \Naturals por Notation
    \DeclareMathOperator \Primes       {\mathbb{P}}                 %Use: \Naturals por Notation
    \DeclareMathOperator \Integers     {\mathbb{Z}}                 %Use: \Integers por Notation
    \DeclareMathOperator \Racionals    {\mathbb{Q}}                 %Use: \Racionals por Notation
    \DeclareMathOperator \Reals        {\mathbb{R}}                 %Use: \Reals por Notation
    \DeclareMathOperator \Complexs     {\mathbb{C}}                 %Use: \Complex por Notation
    \DeclareMathOperator \GenericField {\mathbb{F}}                 %Use: \Complex por Notation

    % === LINEAL ALGEBRA & VECTORS =============
    \DeclareMathOperator \LinealTransformation {\mathcal{T}}        %Use: \LinealTransformation for a cool T
    \newcommand{\Mag}[1]{\left| #1 \right|}                         %Use: \Mag {Info} 
    \newcommand{\bVec}[1]{\mathbf{#1}}                              %Use for bold type of vector
    \newcommand{\lVec}[1]{\overrightarrow{#1}}                      %Use for a long arrow over a vector
    \newcommand{\uVec}[1]{\boldsymbol{\hat{\textbf{#1}}}}           %Use: Unitary Vector Example: $\uVec{i}

    % === WRAPPERS FOR COLUMN VECTOR ===
    \newcommand{\pVector}[1]                                        %Use: \pVector {Matrix Notation} use parentesis
        { \ensuremath{\begin{pmatrix}#1\end{pmatrix}} }             %Example: \pVector{a\\b\\c} or \pVector{a&b&c} 
    \newcommand{\lVector}[1]                                        %Use: \lVector {Matrix Notation} use a abs 
        { \ensuremath{\begin{vmatrix}#1\end{vmatrix}} }             %Example: \lVector{a\\b\\c} or \lVector{a&b&c} 
    \newcommand{\bVector}[1]                                        %Use: \bVector {Matrix Notation} use a brackets 
        { \ensuremath{\begin{bmatrix}#1\end{bmatrix}} }             %Example: \bVector{a\\b\\c} or \bVector{a&b&c} 
    \newcommand{\Vector}[1]                                         %Use: \Vector {Matrix Notation} no parentesis
        { \ensuremath{\begin{matrix}#1\end{matrix}} }               %Example: \Vector{a\\b\\c} or \Vector{a&b&c}

    % === MAKE MATRIX BETTER  =========
    \makeatletter                                                   %Example: \begin{matrix}[cc|c]
    \renewcommand*\env@matrix[1][*\c@MaxMatrixCols c] {             %WTF! IS THIS
        \hskip -\arraycolsep                                        %WTF! IS THIS
        \let\@ifnextchar\new@ifnextchar                             %WTF! IS THIS
        \array{#1}                                                  %WTF! IS THIS
    }                                                               %WTF! IS THIS
    \makeatother                                                    %WTF! IS THIS

    % ===== TRIGONOMETRIC FUNCTIONS  ==========
    \newcommand{\Cos}[1]{\cos\Wrap{#1}}                             %Simple wrappers
    \newcommand{\Sin}[1]{\sin\Wrap{#1}}                             %Simple wrappers
    \newcommand{\Tan}[1]{tan\Wrap{#1}}                              %Simple wrappers
    
    \newcommand{\Sec}[1]{sec\Wrap{#1}}                              %Simple wrappers
    \newcommand{\Csc}[1]{csc\Wrap{#1}}                              %Simple wrappers
    \newcommand{\Cot}[1]{cot\Wrap{#1}}                              %Simple wrappers

    % === COMPLEX ANALYSIS TRIG ===
    \newcommand \Cis[1]  {\Cos{#1} + i \Sin{#1}}                    %Use: \Cis for cos(x) + i sin(x)
    \newcommand \pCis[1] {\Wrap{\Cis{#1}}}                          %Use: \pCis for the same ut parantesis
    \newcommand \bCis[1] {\Brackets{\Cis{#1}}}                      %Use: \bCis for the same to Brackets


    % === CALCULUS ==========================
    
    % ====== TRANSFORMS =====
    \newcommand{\FourierT}[1]{\mathscr{F} \left\{ #1 \right\} }     %Use: \FourierT {Funtion}
    \newcommand{\InvFourierT}[1]{\mathscr{F}^{-1}\left\{#1\right\}} %Use: \InvFourierT {Funtion}

    % ====== DERIVATE ======
    \newcommand \MiniDerivate[1][x] {\dfrac{d}{d #1}}               %Use: \MiniDerivate for simple use
    \newcommand \Derivate[2]                                        %Complete Derivate -- [f(x)][x]
        {\dfrac{d \; #1}{d #2}}                                     %Use: \Partial for simple use
    
    \newcommand \MiniUpperDerivate[2]                               %Mini Derivate High Orden Derivate -- [x][1]
        {\dfrac{d^{#2}}{d#1^{#2}}}                                  %Mini Derivate High Orden Derivate
    \newcommand \UpperDerivate[3]                                   %Complete High Orden Derivate -- [f(x)][x][1]
        {\dfrac{d^{#3} \; #1}{d#2^{#3}}}                            %Use: \UpperDerivate for simple use
    
    \newcommand \MiniPartial[1][x] {\dfrac{\partial}{\partial #1}} %Use: \MiniDerivate for simple use
    \newcommand \Partial[2]                                        %Complete Derivate -- [f(x)][x]
        {\dfrac{\partial \; #1}{\partial #2}}                      %Use: \Partial for simple use
    
    \newcommand \MiniUpperPartial[2]                                %Mini Derivate High Orden Derivate -- [x][1] 
        {\dfrac{\partial^{#2}}{\partial #1^{#2}}}                   %Mini Derivate High Orden Derivate
    \newcommand \UpperPartial[3]                                    %Complete High Orden Derivate -- [f(x)][x][1]
        {\dfrac{\partial^{#3} \; #1}{\partial#2^{#3}}}              %Use: \UpperDerivate for simple use

    \DeclareMathOperator \Evaluate  {\Big|}                         %Use: \Evaluate por Notation

    
    % =====  GENERAL COLOR  ==================
    \definecolor{TealMD}{HTML}{009688}                              %Use: Color :D        
    \definecolor{IndigoMD}{HTML}{3F51B5}                            %Use: Color :D
    \definecolor{Green100MD}{HTML}{DCEDC8}                          %Use: Color :D
    \definecolor{Blue300MD}{HTML}{64B5F6}                           %Use: Color :D
    \definecolor{DeepPurpleMD}{HTML}{673AB7}                        %Use: Color :D
    \definecolor{BlueGrey100MD}{HTML}{CFD8DC}                       %Use: Color :D
    \definecolor{BlueGrey800MD}{HTML}{37474F}                       %Use: Color :D
    \definecolor{BlueGrey200MD}{HTML}{B0BEC5}                       %Use: Color :D
    \definecolor{Lime300MD}{HTML}{E6EE9C}                           %Use: Color :D

    \newenvironment{ColorText}[1]{                                  %Use: \begin{ColorText}
        \leavevmode\color{#1}\ignorespaces}                         %That's is!

    % =====  CODE EDITOR =========
    \lstdefinestyle{CompilandoStyle} {                              %This is Code Style
        backgroundcolor=\color{BlueGrey800MD},                      %Background Color  
        basicstyle=\tiny\color{white},                              %Font color
        commentstyle=\color{BlueGrey200MD},                         %Comment color
        stringstyle=\color{Lime300MD},                              %String color
        keywordstyle=\color{Blue300MD},                             %keywords color
        numberstyle=\tiny\color{TealMD},                            %Size of a number
        frame=shadowbox,                                            %Adds a frame around the code
        breakatwhitespace=true,                                     %Style   
        showstringspaces=false,                                     %Hate those spaces                  
        breaklines=true,                                            %Style                   
        keepspaces=true,                                            %Style                   
        numbers=left,                                               %Style                   
        numbersep=10pt,                                             %Style 
        xleftmargin=\parindent,                                     %Style 
        tabsize=4,                                                  %Style
        inputencoding=utf8/latin1                                   %Allow me to use special chars
    }
 
    \lstset{style=CompilandoStyle}                                  %Use this style




% =====================================================
% ============        COVER PAGE       ================
% =====================================================
\begin{document}
\begin{titlepage}
    
    % ============ TITLE PAGE STYLE  ================
    \definecolor{TitlePageColor}{cmyk}{1,.60,0,.40}                 %Simple colors
    \definecolor{ColorSubtext}{cmyk}{1,.50,0,.10}                   %Simple colors
    \newgeometry{left=0.25\textwidth}                               %Defines an Offset
    \pagecolor{TitlePageColor}                                      %Make it this Color to page
    \color{white}                                                   %General things should be white
    \newcommand{\Github}{https://github.com/compilandoconocimiento} %The general Parte

    % ===== MAKE SOME SPACE =========
    \vspace                                                         %Give some space
    \baselineskip                                                   %But we need this to up command

    % ============ NAME OF THE PROJECT  ============
    \makebox[0pt][l]{\rule{1.3\textwidth}{3pt}}                     %Make a cool line
    
    \href{\Github}                                                  %Link to project
    {\textbf{\textsc{\Huge Compilando Conocimiento}}}\\[2.7cm]      %Name of project   

    % ============ NAME OF THE BOOK  ===============
    \href{\Github/LibroAnalisisVectorial}                           %Link to Author
    {\fontsize{55}{66}\selectfont                                   %Set size
        \textbf{Análisis Vectorial}}\\[0.5cm]                       %Name of the book
    \textcolor{ColorSubtext}{\textsc{\Huge Cálculo}}                %Name of the general theme
    
    \vfill                                                          %Fill the space
    
    % ============ NAME OF THE AUTHOR  =============
    \href{https://github.com/alaneos777}                            %Link to Author
    {\LARGE \textsf{Alan Enrique Ontiveros Salazar}}                %Author

    % ===== MAKE SOME SPACE =========
    \vspace                                                         %Give some space
    \baselineskip                                                   %But we need this to up command
    
    {\large \textsf{Enero 2018}}                                    %Date

\end{titlepage}


% =====================================================
% ==========      RESTORE TO DOCUMENT      ============
% =====================================================
\restoregeometry                                                    %Restores the geometry
\nopagecolor                                                        %Use to restore the color to white






% =====================================================
% ========                INDICE              =========
% =====================================================
\tableofcontents{}
\label{sec:Index}

\clearpage



% //////////////////////////////////////////////////////////////////////////////////////////////////////////
% ////////////////////////////////         INTRODUCCION A LOS VECTORES     /////////////////////////////////
% //////////////////////////////////////////////////////////////////////////////////////////////////////////
\part{Introducción a los Vectores sobre $\Reals$}


    % ===============================================================================
    % ===================            CONCEPTOS BASICOS         ======================
    % ===============================================================================
    \chapter{Conceptos Básicos}
    

        % =========================================================
        % ==========      DEFINICION DE ESCALAR      ==============
        % =========================================================
        \clearpage
        \section{Definición de Escalar}

            Definiremos a los escalares como elementos de $\Reals$, es decir, cualquier número de
            la recta real.
            Reciben ese nombre porque al ser multiplicados por un vector, como veremos más adelante,
            lo pueden aumentar o disminuir de tamaño, es decir, los escalan.

            Son usados para describir cantidades que solo dependen de un número (y posiblemente una
            unidad en Física por ejemplo) para ser descritas completamente, por ejemplo, masa,
            volumen, temperatura, longitud, etc.


        % =========================================================
        % ==========      DEFINICION DE ESCALAR      ==============
        % =========================================================
        \vspace{1em}
        \section{Definición de Vector}
        
            Probablemente el concepto de vector es el que más definiciones tiene dependiendo de qué
            punto de vista se estudien.

            \textbf{Aquí solo veremos cómo definirlos sobre el plano de $\Reals^2$ y el espacio de $\Reals^3$}.

            También existen muchas formas de escribirlos, aquí usaremos de manera general una flecha
            arriba de la variable: $\vec{a}$, aunque también nos dará la gana y podemos poner la
            variable en negritas: $\bVec{a}$

            % =========================================================
            % ==========     PUNTO DE VISTA GEOMETRICO     ============
            % =========================================================
            \subsection{Punto de Vista Geométrico}
            
                Podemos extender el concepto de un punto en el espacio y definir a un vector
                como la flecha que apunta desde el origen hasta ese punto.

                De esta forma vemos que un vector tiene \textbf{magnitud} (la longitud desde
                el origen hasta el punto), \textbf{dirección} (es decir la recta que pasa por
                el origen y ese punto) y \textbf{sentido} (hacia dónde apunta la flecha).
                
            % =========================================================
            % ==========     PUNTO DE VISTA ALGEBRAICO     ============
            % =========================================================
            \subsection{Punto de Vista Algebráico}
            
                Un vector $\vec{a}$ es un elemento de $\Reals^2$ o de $\Reals^3$, y escribimos
                $\vec{a} = (a_1, a_2, a_3)$, donde $a_1, a_2, a_3 \in \Reals$ son sus
                \textbf{coordenadas} o \textbf{componentes}.

                Por lo tanto no es más que un simple par o terna ordenada de números. 
                De una manera similiar (aunque no lo veremos ahora, podemos ampliar la idea de vectores
                sobre $\Reals$ (o sobre cualquier campo) como un tupla de n-reales).

                Si pasa que $a_3 = 0$, simplemente podemos escribir $(a_1, a_2)$ para un vector en el plano.
                De forma similar, las propiedades que se cumplan para un vector en $\Reals^3$ se
                cumplen para vectores en $\Reals^2$ ignorando la tercera componente.

            
        % =========================================================
        % =======  DIFERENCIA ENTRE PUNTO Y VECTOR     ============
        % =========================================================
        \clearpage
        \section{Relaciones entre Puntos y Vectores}
        
            En esencia un vector y un punto son lo mismo, pero un punto solo indica una posición
            en el espacio, mientras que un vector indica un \textbf{desplazamiento}.
            Lo veremos a continuación.
            
            % ========================================
            % =======  VECTOR POSICION    ============
            % ========================================
            \subsection{Vector Posición}
            
                Dado un punto $P$, definimos al vector posición de $P$ respecto de un origen $O$ como
                el vector $\lVec{OP}$, que tendrá las mismas coordenadas del punto $P$.
            
            % ========================================
            % =======  VECTOR DESPLAZAMIENTO    ======
            % ========================================
            \subsection{Vector Desplazamiento}
            
                Dados dos puntos $P$ y $Q$, definimos al vector desplazamiento de $P$ a $Q$ como el
                vector $\lVec{PQ}$, en donde el origen de la flecha está en $P$ y la punta en $Q$.
                De esta forma vemos que los vectores no necesariamente comienzan en el origen,
                sino en donde queramos. Es muy importante comprender y recordar esto a lo largo
                de este librito.

                Ahora, supón 2 puntos $P = (x_1, y_1, z_2) \in \Reals^2$ y $Q = (x_1, y_1, z_2) \in \Reals^3$.

                Entonces tenemos que las coordenadas del vector $\lVec{PQ}$ se puede ver como:\\
                $PQ = (x_2 - x_1, y_2 - y_1, z_2 - z_1)$









    % ===============================================================================
    % ===================            ALGEBRA VECTORIAL         ======================
    % ===============================================================================
    \chapter{Álgebra Vectorial}
            
        

        % =========================================================
        % ============   OPERACIONES BASICAS     ==================
        % =========================================================
        \clearpage
        \section{Operaciones Básicas}
            
            % ==============================================
            % ========   IGUALDAD DE VECTORES    ===========
            % ==============================================
            \subsection{Igualdad de Vectores}

                \begin{Definition}[Igualdad de 2 Vectores]
                    \label{DefIgualdadVectores}
                    Decimos que dos vectores son iguales si y solo si sus componentes
                    correspondientes son iguales.
                \end{Definition}

                Es decir, si tenemos $\vec{a} = (a_1, a_2, a_3)$ y $\vec{b} = (b_1, b_2, b_3)$,
                entonces $\vec{a} = \vec{b}$ si y solo si $a_1 = b_1$, $a_2 = b_2$ y $a_3 = b_3$.

            
            % ==============================================
            % ========       SUMA Y RESTA        ===========
            % ==============================================
            \subsection{Suma y Resta}
            
                \begin{Definition}[Suma de Vectores]
                    \label{DefSumaVectores}

                    Decimos que $\vec{a}+\vec{b}$ es la suma de $\vec{a}$ con $\vec{b}$ si y solo si 
                    cada componente de $\vec{a}+\vec{b}$ es la suma de los correspondientes componentes
                    de $\vec{a}$ y $\vec{b}$.
                    O sea, simplemente sumamos componente a componente.

                \end{Definition}

                Sean $\vec{a} = (a_1, a_2, a_3) \in \Reals^3$ y $\vec{b}=(b_1, b_2, b_3) \in \Reals^3$.

                Entonces decimos que:
                \begin{equation*}
                    \vec{a} + \vec{b} := (a_1 + b_1, a_2 + b_2, a_3 + b_3)
                \end{equation*}
            
                \textbf{Geométricamente}, para sumar $\vec{a}$ con $\vec{b}$ colocamos el principio de
                $\vec{b}$ junto a la punta de $\vec{a}$. El vector $\vec{a} + \vec{b}$ será el vector
                que comienza en donde comienza $\vec{a}$ y termina en donde termina $\vec{b}$.
                
                \vspace{2em}

                \begin{Definition}[Resta de Vectores]
                    \label{DefRestaVectores}

                    Decimos que $\vec{a}-\vec{b}$ es la resta de $\vec{a}$ menos $\vec{b}$ si y solo si 
                    cada componente de $\vec{a}-\vec{b}$ es la resta de los correspondientes componentes
                    de $\vec{a}$ y $\vec{b}$.
                    O sea, simplemente restamos componente a componente.
                    
                \end{Definition}

                Sean $\vec{a} = (a_1, a_2, a_3) \in \Reals^3$ y $\vec{b}=(b_1, b_2, b_3) \in \Reals^3$.

                Entonces decimos que:
                \begin{equation*}
                    \vec{a} - \vec{b} := (a_1 - b_1, a_2 - b_2, a_3 - b_3)
                \end{equation*}
            
                Con esta definición, podemos decir que el vector desplazamiento del punto $P$ al punto $Q$
                es el vector $\lVec{PQ} = \lVec{OQ} - \lVec{OP}$.
                
            % ==============================================
            % ===     MULTIPLICACION POR ESCALAR       =====
            % ==============================================
            \clearpage
            \subsection{Multiplicación por Escalar y Propiedades}

                \begin{Definition}[Producto de un Vector por un Escalar]
                    \label{DefProductoVectorEscalar}

                    Decimos que $k \vec{a}$ es el producto escalar de $a$ con $k$
                    si y solo si cada componente de $k \vec{a}$ es la el producto del correspondiente componente
                    de $\vec{a}$ multiplicada por $k$.

                    Es decir, solo multiplicamos cada componente (que son escalares) por el escalar $k$.

                \end{Definition}

                Sea $\vec{a}=(a_1, a_2, a_3) \in \Reals^3$ un vector y $k \in \Reals$ un escalar.
                
                Entonces decimos que:
                \begin{equation*}
                    k\vec{a} = (ka_1, ka_2, ka_3)
                \end{equation*}

                \textbf{Geométricamente}, multiplicar un vector por un escalar es de agrandarlo o reducirlo pero
                sin cambiar su dirección (su sentido se invierte si $k < 0$, se queda igual si $k > 0$ y
                obtenemos el cero vector si $k = 0$).



        % =========================================================
        % ====    PROPIEDADES DE LAS OPERACIONES BASICAS     ======
        % =========================================================
        \clearpage
        \section{Propiedades de Operaciones}
        
            Las operaciones anteriores cumplen con las siguientes propiedades, donde
            $\vec{a},\vec{b}, \vec{c} \in \Reals^3$ y $\alpha, \beta \in \Reals$.
            Vemos que todas se heredan de las ya conocidas propiedades de los números reales:

            \begin{itemize}
                
                \item \textbf{Conmutativa:} 
                    $\vec{a}+\vec{b} = \vec{b}+\vec{a}$.
                
                    % ======== DEMOSTRACION ========
                    \begin{SmallIndentation}[1em]
                        \textbf{Demostración:}

                        Se sigue inmediatamente de la propiedad conmutativa de los números reales:
                        \begin{align*}
                            \vec{a} + \vec{b} 
                                &= (a_1, a_2, a_3) + (b_1, b_2, b_3)  &&\Remember{Expresar en coordenadas}              \\
                                &= (a_1 + b_1, a_2 + b_2, a_3 + b_3)  &&\Remember{Definición de suma de vectores}       \\
                                &= (b_1 + a_1, b_2 + a_2, b_3 + a_3)  &&\Remember{Propiedad conmutativa de los reales}  \\
                                &= (b_1, b_2, b_3) + (a_1, a_2, a_3)  &&\Remember{Definición de suma a la inversa}      \\
                                &= \vec{b} + \vec{a}                  &&\Remember{Volvemos a armar a los vectores}
                        \end{align*}

                    \end{SmallIndentation}
                
                \item \textbf{Asociativa:} 
                    $\vec{a} + \Wrap{\vec{b}+\vec{c}} = \Wrap{\vec{a} + \vec{b} } + \vec{c}$
                    
                    % ======== DEMOSTRACION ========
                    \begin{SmallIndentation}[1em]
                        \textbf{Idea de Demostración:}
                    
                            Exactamente la misma que la anterior, solo que usando la propiedad
                            asociativa en los reales.

                            Como ambas expresiones son iguales, podemos escribir sin ambigüedad que:\\
                            $\vec{a} + \Wrap{\vec{b} + \vec{c}} 
                                = \Wrap{\vec{a} + \vec{b}} + \vec{c} 
                                = \vec{a} + \vec{b} + \vec{c}$

                    \end{SmallIndentation}
                
                \item \textbf{Neutro Aditivo:} 
                    Existe $\vec{0} \in \Reals^3$ (el cero vector) tal que $\vec{a} + \vec{0} = \vec{a}$.

                    % ======== DEMOSTRACION ========
                    \begin{SmallIndentation}[1em]
                        \textbf{Ideas}:
                        
                        ¿Quién crees que sea ese cero vector? Exacto, $\vec{0} = (0, 0, 0)$
                        \hfill
                        \textbf{Este vector es el único que no tiene una dirección ni un sentido bien definidos}.
                        
                    \end{SmallIndentation}
                
                \item \textbf{Inverso Aditivo:}
                    Existe $-\vec{a} \in \Reals^3$ tal que $\vec{a} + \Wrap{-\vec{a}} = \vec{0}$.

                    % ======== DEMOSTRACION ========
                    \begin{SmallIndentation}[1em]
                        \textbf{Ideas}:
                        
                        Justamente las coordenadas de $-\vec{a}$ son los inversos aditivos en los
                        reales de sus coordenadas: $-\vec{a} = (-a_1, -a_2, -a_3)$

                    \end{SmallIndentation}
                
                \item \textbf{Distributiva sobre Escalares:} 
                    $(\alpha + \beta) \vec{a} = \alpha\vec{a} + \beta\vec{a}$
                
                    % ======== DEMOSTRACION ========
                    \begin{SmallIndentation}[1em]
                        \textbf{Demostración:}
                        \begin{align*}
                            (\alpha + \beta) \vec{a}
                                &= (\alpha + \beta)(a_1, a_2, a_3)                  
                                        &&\Remember{Expresar en coordenadas}                                    \\
                                &= \Wrap{(\alpha + \beta)a_1, (\alpha + \beta)a_2, (\alpha + \beta)a_3} 
                                        &&\Remember{Definición de producto por escalar}                         \\
                                &= (\alpha a_1 + \beta a_1, \alpha a_2 + \beta a_2, \alpha a_3 + \beta a_3)  
                                        &&\Remember{Propiedad distributiva en los reales}                       \\
                                &= (\alpha a_1, \alpha a_2, \alpha a_3) + (\beta a_1, \beta a_2, \beta a_3)
                                        &&\Remember{Definición de suma a la inversa}                            \\
                                &= \alpha (a_1, a_2, a_3) + \beta(a_1, a_2, a_3)             
                                        &&\Remember{Definición de producto a la inversa}                        \\
                                &= \alpha \vec{a} + \beta \vec{a}    
                                        &&\Remember{Volvemos a armar a los vectores}
                        \end{align*}

                    \end{SmallIndentation}


                \clearpage


                
                \item \textbf{Distributiva sobre Vectores:} 
                    $\alpha {\vec{a} + \vec{b}} = \alpha \vec{a} + \alpha \vec{b}$

                    % ======== DEMOSTRACION ========
                    \begin{SmallIndentation}[1em]
                        \textbf{Idea de Demostración:} 
                        Usa la definición de suma de vectores y la de multiplicación por escalar,
                        debería de quedarte al primer intento.
                    \end{SmallIndentation}
                
                \item \textbf{Asociativa sobre Escalares:}
                    $\alpha \Wrap{ \beta \vec{a} } = (\alpha \beta)\vec{a} = \beta \Wrap{\alpha\vec{a}}$
                
                    % ======== DEMOSTRACION ========
                    \begin{SmallIndentation}[1em]
                        \textbf{Idea de Demostración:}
                            Las mismas técnicas que las anteriores. Ya no quiero hacer más demostraciones :o
                    \end{SmallIndentation}

            \end{itemize}
        
            Todas las propiedades anteriores se pueden generalizar perfectamente a vectores en cualquier
            dimensión, es decir, que pertenezcan a $\Reals^n$.
            


        % =========================================================
        % ==============          MAGNITUD            =============
        % =========================================================
        \clearpage
        \section{Magnitud}
        
            \begin{Definition}[Magnitud de un Vector]
                Sea $\vec{a} \in \Reals^3$. Definimos a la magnitud o al módulo de $\vec{a}$ como:
                \begin{equation*}
                    \Abs{\vec{a}} := \sqrt{(a_1)^2 + (a_2)^2 + (a_3)^2}
                \end{equation*}
            \end{Definition}
    
            Geométricamente es la distancia del origen a la punta del vector debido al teorema
            de Pitágoras. A veces se usa la notación $\abs{\vec{a}}$ o simplemente el vector
            sin flecha $a$ para referirse a su magnitud, pero aquí usaremos dobles barras.
        

            % =========================================================
            % ==============       PROPIEDADES            =============
            % =========================================================
            \vspace{2em}
            \subsubsection{Propiedades}

                \begin{itemize}
                    
                    \item Si $k$ es un escalar y $\vec{a}$ es un vector entonces:\\ 
                        $\Abs{k\vec{a}} = \abs{k} \Abs{\vec{a}}$
                            
                        \begin{SmallIndentation}[1em]
                            \textbf{Demostración:}

                            Es decir, la magnitud del producto de un escalar por un vector es
                            igual al valor absoluto del escalar por la magnitud del vector,
                            por lo que de alguna forma podemos \Quote{sacar} el escalar.

                            \begin{align*}
                                \Abs{k\vec{a}}
                                    &= \Abs{k(a_1, a_2, a_3)}                       &&\mbox{Representación en Coordenadas}             \\
                                    &= \Abs{(ka_1, ka_2, ka_3)}                     &&\mbox{Definición de multiplicación por escalar}  \\
                                    &= \sqrt{(ka_1)^2 + (ka_2)^2 + (ka_3)^2}        &&\mbox{Definición de magnitud}                    \\
                                    &= \sqrt{k^2(a_1)^2 + k^2(a_2)^2 + k^2(a_3)^2}  &&\mbox{Propiedad de los exponentes en los reales} \\
                                    &= \sqrt{k^2(a_1^2 + a_2^2 + a_3^2)}            &&\mbox{Factorización}                             \\
                                    &= \sqrt{k^2}\sqrt{a_1^2 + a_2^2 + a_3^2}       &&\mbox{Se distribuye sobre el producto de reales} \\
                                    &= \abs{k} \Abs{\vec{a}}                        &&\mbox{Propiedad de raíz y definición de magnitud}
                            \end{align*}

                        \end{SmallIndentation}

                \end{itemize}




        % =========================================================
        % ==============          MAGNITUD            =============
        % =========================================================
        \clearpage
        \section{Vector Unitario}

            \begin{Definition}[Vector Unitario]
                Si un vector $\vec{v}$ cumple que $\Abs{\vec{v}} = 1$, decimos que es un
                \emph{vector unitario} y usualmente se denota como $\hat{v}$.
            \end{Definition}

            Recordando a la propiedad de que $\Abs{k\vec{a}} = \abs{k} \Abs{\vec{a}}$ lo anterior
            nos motiva a que, dado un vector cualquiera $\vec{v} \neq \vec{0}$, queramos obtener
            su equivalente unitario $\hat{v}$, es decir, el vector con su misma dirección y sentido
            pero con magnitud 1.

            Dicho proceso se conoce como normalización.

            % =========================================================
            % ==================    NORMALIZACION         =============
            % =========================================================
            \subsection{Normalización}
                
                \begin{Theorem}[Normalización]
                    Sea $\vec{v}$ un vector entonces podemos obtener su vector unitario como:
                    \begin{equation*}
                        \hat{v} = \dfrac{1}{ \Abs{\vec{v}}} \; \vec{v}
                    \end{equation*}
                \end{Theorem}
            
                \begin{SmallIndentation}[1em]
                    \textbf{Demostración:}

                    Es fácil ver que la magnitud del vector propuesto es 1, usando lo que ya sabemos:
                    \begin{align*}
                        \Abs{\hat{v}} 
                            &= \Abs{\dfrac{1}{\Abs{\vec{v}}} \vec{v}}               \\
                            &= \abs{\dfrac{1}{\Abs{\vec{v}}}} \Abs{\vec{v}}         \\
                            &= \dfrac{1}{\Abs{\vec{v}}} \Abs{\vec{v}}               \\
                            &= 1
                    \end{align*}

                    Y obviamente $\hat{v}$ tiene la misma dirección y sentido que $\vec{v}$,
                    pues $\frac{1}{\Abs{\vec{v}}}$ siempre es positivo y está multiplicando a $\vec{v}$.

                \end{SmallIndentation}
            
            % =========================================================
            % ==================    NORMALIZACION         =============
            % =========================================================
            \clearpage
            \subsection{Representación en Vectores Unitarios}
        
                \begin{Definition}[Vectores Unitarios Canónicos]
                    
                    Introducimos a los siguientes vectores:
                    \begin{align*}
                        \hat{i} &= (1, 0, 0) \MegaSpace \hat{j} = (0, 1, 0) \MegaSpace \hat{k} = (0, 0, 1)
                    \end{align*}

                \end{Definition}
            
                \subsubsection{¿Para qué nos sirve tener a esos vectores?}

                    Simple, para poder escribir cualquier vector $\vec{a} \in \Reals$ como combinación
                    lineal de ellos en vez de usar la tupla.
                    Veamos cómo:
                    \begin{SmallIndentation}[1em]
                        \begin{align*}
                            \vec{a}
                                &= (a_1, a_2, a_3)                               &&\mbox{Representación en coordenadas}        \\
                                &= (a_1 + 0 + 0, 0 + a_2 + 0, 0 + 0 + a_3)       &&\mbox{Sumamos ceros convenientemente}       \\
                                &= (a_1, 0, 0) + (0, a_2, 0) + (0, 0, a_3)       &&\mbox{Definición de suma}                   \\
                                &= a_1(1, 0, 0) + a_2(0, 1, 0) + a_3(0, 0, 1)    &&\mbox{Factorización del escalar}            \\
                                &= \mathbf{a_1\hat{i} + a_2\hat{j} + a_3\hat{k}} &&\mbox{Definición de los vectores canónicos}
                        \end{align*}
                    \end{SmallIndentation}
                
                    Es fácil ver que los tres vectores propuestos son unitarios, y la expresión anterior
                    quiere decir que nos estamos desplazando $a_1$ unidades en dirección al eje $\mathbf{x}$,
                    $a_2$ unidades en dirección al eje $\mathbf{y}$ y $a_3$ unidades en dirección al eje
                    $\mathbf{z}$.

                    De esta forma, el desplazamiento total será justamente el vector $\vec{a}$.
                    



        % =========================================================
        % =====   DEPENDENCIA E INDEPENDENCIAL LINEAL      ========
        % =========================================================
        \clearpage
        \section{Dependencia e independencia lineal}
            
            Este es un concepto importante antes de pasar a otras operaciones que podemos
            hacer con los vectores.
            
            \begin{Definition}
                Sean $k_1, k_2, \ldots, k_n \in \Reals$ escalares y $\vec{v_1}, \vec{v_2}, \ldots, \vec{v_n} \in \Reals^3$
                vectores.

                Decimos que dichos vectores son \textbf{linealmente independientes} si y solo si:
                \begin{equation*}
                    \sum_{i=1}^{n} k_i \vec{a_i} = \vec{0}
                        \Space
                        \text{implica}
                        \Space
                    k_1 = k_2 = \cdots = k_n = 0
                \end{equation*}
                
                En caso contrario decimos que los vectores son \textbf{linealmente dependientes}.
            \end{Definition}

            % =========================================================
            % =================     IDEAS IMPORTANTES     =============
            % =========================================================
            \subsection{Ideas Importantes}
        
                La definición anterior es realmente interesante, porque si la tomamos a la inversa,
                es decir, asumiendo que todos los escalares $k_1, k_2, \ldots, k_n$ valen cero,
                entonces la combinación lineal de los vectores siempre daría $\vec{0}$, lo cual
                no es de mucha utilidad.
            
                Ahora veamos cómo entenderla. Si tenemos $n$ vectores, que son $\vec{v_1}, \vec{v_2}, \ldots, \vec{v_n}$,
                son linealmente independientes (l.i. para abreviar) si la única forma de obtener el cero vector $\vec{0}$
                al multiplicarlos cada uno por un escalar y luego sumar todo es que dichos escalares sean todos 0.
                Si somos capaces de encontrar algunos otros escalares para esta tarea y el resultado también es $\vec{0}$,
                los vectores son linealmente dependientes (l.d.).
                
                Una consecuencia de lo anterior es que si los vectores son l.d., entonces alguno de ellos se puede
                escribir como combinación lineal de los otros. Geométricamente, tomando dichos otros vectores
                multiplicados por algún escalar como suma de desplazamientos, llegaremos a obtener el vector inicial.
                Si los vectores fueran l.i. esto no sería posible, nunca podríamos obtener un vector como la suma de
                desplazamientos de los otros.
            
                \begin{Theorem}
                    Sean $\vec{v_1}, \vec{v_2}, \ldots, \vec{v_n} \in \Reals^3$ vectores linealmente dependientes.

                    Entonces existe $j$ tal que $1 \leq j \leq n$ y escalares $l_1, l_2, \ldots, l_{n-1} \in \Reals$
                    tales que:
                    \begin{equation*}
                        \vec{v_j} = \sum_{i=1}^{n-1} l_i \vec{v_i}   
                    \end{equation*}
                \end{Theorem}
        
                \begin{SmallIndentation}[1em]
                    \textbf{Idea de la Demostración:} 
                        Como los vectores son l.d., entonces tiene que haber algún escalar
                        tal que $k_j \neq 0$. Encuentra el vector $\vec{v_j}$ y simplemente
                        despéjalo, eso será posible pues su escalar es distinto de cero.
                \end{SmallIndentation}
            

            % =========================================================
            % =================     IDEAS IMPORTANTES     =============
            % =========================================================
            \subsection{¿Cómo saber si mi conjunto de vectores es l.i. o l.d.?}
            
                Sigamos la definición, propongamos los escalares $k_1, k_2, \ldots, k_n \in \Reals$
                tales que $\sum_{i=1}^{n} k_i \vec{a_i} = \vec{0}$.

                Esto nos llevará a un sistema de $n$ ecuaciones lineales luego de igualar las componente
                del vector resultante con 0. Si logramos demostrar que dicho sistema tiene como \textbf{única}
                solución $k_1 = k_2 = \cdots k_n = 0$, los vectores son l.i.

                Si aparte de esa encontramos otra solución (de hecho si hay más que la solución trivial
                habrá infinitas soluciones), los vectores son l.d.
        


















        \clearpage
        \section{Productos entre vectores}
        
            \subsection{Producto punto}
            
                \subsubsection{Ángulo entre vectores}
                
                \subsubsection{Proyección de un vector sobre otro}
                
                \subsubsection{Desigualdad de Cauchy-Schwarz}
                
                \subsubsection{Desigualdad del triángulo}
            
            \subsection{Producto cruz}
            
                \subsubsection{Área de un paralelogramo}
            
            \subsection{Producto triple}
            
                \subsubsection{Volumen de un paralelepípedo}
                
            \subsection{Propiedades útiles}
            
    \chapter{Aplicaciones a la geometría}
        
        \section{Ecuación del plano}
            
        \section{Ecuación de la recta}
        
        \section{Ecuación de la esfera}
            
        \section{Distancia punto-recta y punto-plano}
        
        \section{Rotaciones en el espacio}
        
        \section{Demostraciones geométricas mediante vectores}
            


\part{Cálculo diferencial vectorial}

    \chapter{Funciones de varias variables}
    
        \section{Representación como superficies}
        
            \subsection{Curvas de nivel y de contorno}
            
        \section{Límites}
            
            \subsection{Definición intuitiva}
            
            \subsection{Definición formal}
            
        \section{Continuidad}
        
        \section{Derivadas parciales}
        
            \subsection{Plano tangente a una superficie}
            
            \subsection{Diferenciabilidad}
            
            \subsection{Derivadas de orden superior}
            
                \subsubsection{Teorema de Clairaut}
                
        \section{Gradiente}
                
        \section{Regla de la cadena}
        
            \subsection{Diferencial total}
            
        \section{Derivada direccional}
        
        \section{Puntos críticos}
        
            \subsection{Máximos, mínimos y puntos silla}
            
            \subsection{Criterio del hessiano}
            
        \section{Multiplicadores de Lagrange}


    \chapter{Funciones vectoriales}
    
        \section{Curvas en forma paramétrica}
        
            \subsection{Reglas de derivación}
            
            \subsection{Velocidad y aceleración}
        
            \subsection{Longitud de arco}
            
            \subsection{Parametrización por longitud de arco}
            
            \subsection{Geometría diferencial}
                
                \subsubsection{Vector tangente, normal y binormal}
                
                \subsubsection{Curvatura y torsión}
                
                \subsubsection{Velocidad y aceleración}
                
                \subsubsection{Ecuaciones de Frenet-Serret}
            
        \section{Campos vectoriales}
        
            \subsection{Líneas de campo}
            
            \subsection{Derivadas parciales}
        
        \section{Operador nabla}
        
            \subsection{Gradiente}
            
            \subsection{Divergencia}
            
            \subsection{Rotacional}
            
            \subsection{Laplaciano}
            
            \subsection{Propiedades}



\part{Cálculo integral vectorial}

    \chapter{Integrales multivariable}
    
        \section{Regiones}
        
            \subsection{Regiones del plano y tipos}
            
            \subsection{Regiones del espacio y tipos}
        
        \section{Integrales iteradas}
        
        \section{Integrales dobles}
            
            \subsection{Integración sobre regiones arbitrarias}
            
            \subsection{¿Cómo hallar los límites de integración?}
            
            \subsection{Teorema de Fubini}
        
        \section{Integrales triples}
            
            \subsection{Integración sobre regiones arbitrarias}
            
            \subsection{¿Cómo hallar los límites de integración?}
            
        \section{Cambio de variable en 2 y 3 dimensiones}
        
            \subsection{Transformación de coordenadas}
            
            \subsection{Jacobiano}
            
        \section{Aplicaciones}
        
            \subsection{Valor promedio}
            
            \subsection{Centro de masa}
            
            \subsection{Momento de inercia}
        
    \chapter{Integrales de funciones vectoriales}
    
        \section{Integrales de línea}
        
            \subsection{Función escalar}
            
            \subsection{Función vectorial}
            
            \subsection{Campos conservativos}
            
                \subsubsection{Potencial}
        
        \section{Integrales de superficie}
        
            \subsection{Superficies en forma paramétrica}
            
                \subsubsection{Vector normal}
            
                \subsubsection{Relación con el Jacobiano}
                
                \subsubsection{Cálculo a través del gradiente}
            
            \subsection{Función escalar}
            
            \subsection{Función vectorial}
        
        \section{Integrales de volumen}
        
            \subsection{Regiones del espacio en forma paramétrica}
            
                \subsubsection{Elemento de volumen}
                
                \subsubsection{Relación con el Jacobiano}
        
            \subsection{Función escalar}
            
        \section{Consejos para parametrizar y definir límites}
    
    \chapter{Teoremas de integración}
    
        \section{Teorema de Green}
        
            \subsection{Cálculo de áreas dado el contorno}
        
        \section{Teorema de Stokes}
        
            \subsection{Frontera de una superficie}
        
        \section{Teorema de la divergencia de Gauss}
        
            \subsection{Superficie cerrada}


\part{Coordenadas curvilíneas}

    \chapter{Coordenadas curvilíneas generalizadas}
    
        \section{Transformación de coordenadas}
        
        \section{Sistemas ortogonales}
        
        \section{Vectores unitarios}
        
            \subsection{Factores de escala}
        
        \section{Integración}
        
            \subsection{Elemento de línea}
            
            \subsection{Elemento de longitud de arco}
            
            \subsection{Elemento de área}
            
            \subsection{Elemento de volumen}
            
        \section{Operador nabla}
        
            \subsection{Gradiente}
            
            \subsection{Divergencia}
            
            \subsection{Rotacional}
            
            \subsection{Laplaciano}
            
        \section{Sistemas comunes de coordenadas}
        
            \subsection{Cilíndricas}
            
            \subsection{Esféricas}
            

\end{document}
